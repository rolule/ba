\chapter{Analyse}
In diesem Kapitel werden die in Kapitel 3 vorgestellten Analysen durchgeführt.

\section{Notizen}

Lambda 128mb 5min stress:
req     fails   peak-rps    avg     p95     p99     min     max     ce  mu  duration 
3864    0       26.3        41ms    63      112     23      740     7   71  6-20ms
3873    0       26.7        40ms    61.3    99.7    22      670     9   73  5-12ms

Lambda 128mb 12min stress (/notes):
u   req     fails   p-rps   p-rpm   avg-ms  p95-ms  p99-ms  min-ms  max ce  mu  dur-ms  std
1   235422  0       546     32540   28      36      74      21      509 44  80  2-2     12
2   402468  0       935     54800   40-48   44-51   107-115 23-18   793 84  82  2-15    14-15
3   664391  0       1536    91000   28-37   38-46   72-93   23-30   1k  129 81  2-5     12

Fargate 128cpu 128mb 1task +ELB 12min stress (/notes):
u   req     fails   p-rps   p-rpm   avg-ms  p95-ms  p99-ms  min-ms  max     mu  cpu     std
1   291384  0       676     39800   11      12      41      8       344     8   92      5
2   434568  1 (502) 796     45800   32-41   80-89   147-152 8-13    250-315 8   100     12-28
3   481356  1 (502) 881     45800   59-246  119-239 150-253 8-19    353-354 8   100     44-62
4   488114  5 (502) 987     45400   89-175  138-267 251-357 9-89    457-550 9   100     67-77

Fargate 128cpu 128mb 2task +ELB 12min stress (/notes):
u   req     fails   p-rps   p-rpm   avg-ms  p95-ms  p99-ms  min-ms  max     mu  cpu     std
1      

Fargate 256cpu 512mb 1task stress:
req     fails   peak-rps    peak-rpm    avg-ms  p95-ms  p99-ms  min     max     cpu
880788  0       1982        109000      17-98   19-101  60-154  7-89    151-237 100

Zu untersuchende Aspekte:
1. Wie ist die Performance bei normalem Load oder bei Spike Load?
    H1: Bei Container kein Unterschied, bei Lambda enormer Unterschied
    -> Load Test
    -> Spike Test
    
2. Wie viele Benutzer kann ein Container abdecken bevor die Performance degradiert?
    -> Stress Test
3. Wie ändert sich die Performance bei der Verwendung mehrerer Container Instanzen (Tasks)?
    H2: Doppelt so viele Requests können verarbeitet werden
4. Wie verändert sich die Performance bei einem Use-Case mit mehreren Endpunkten?
    H3: Bei Container kein Unterschied, bei Lambda enormer Unterschied (ähnlich H1)
5. Wie verändert sich die Performance bei größerer CPU und RAM?



Tests
- 128MB Konfiguration
    - Use-Case A
        1. Pipe-Clean
            Lambda
            1 Task -> Annahme: Performance ist für 2 Tasks gleich
        2. Stress Test Container: Wie viele VUs
            1 Task
            2 Tasks
            Lambda: Zum Vergleich
        3. Load Test: wie ist die Response-Time
            Lambda
            1 Task
            2 Tasks
        4. Spike Test
            Lambda
            1 Task
            2 Tasks
            
    - Use-Case B -> Vergleich der Performance für Use-Case mit mehreren Endpunkten
        1. Pipe-Clean
            Lambda
            1 Task
        2. Load Test -> Vergleichen mit Use-Case A
            Lambda
            1 Task
            2 Tasks
        3. Spike Test -> Vergleichen mit Use-Case A
            Lambda
            1 Task
            2 Tasks
        
    - Use-Case C -> Vergleich der Performance für Use-Case mit mehreren Endpunkten
        1. Pipe-Clean
            Lambda
            1 Task
        2. Load Test -> Vergleichen mit Use-Case A und B
            Lambda
            1 Task
            2 Task
        3. Spike Test -> -> Vergleichen mit Use-Case A und B
            Lambda
            1 Task
            2 Task
            
    => Auswertung
        1. Vergleich 
    
    
- 256 MB Konfiguration -> Vergleich Performance mit mehr CPU/RAM