%*******************************************************
% Abstract in English
%*******************************************************
\pdfbookmark[1]{Abstract}{Abstract}


\begin{otherlanguage}{american}
\chapter*{Abstract}
Der Betrieb von Anwendungen in der Cloud erfährt steigende Aufmerksamkeit. Public Cloud Anbieter erlauben es Organisationen, schneller und günstiger ihre Anwendungen und Daten zu provisionieren. Um die Vorteile des Cloud Deployments nutzen zu können, haben sich das Serverless Paradigma und die Containerisierung als relevante Entwicklungen herausgebildet. Während Serverless-Anwendungen fast vollständig von den Cloud Anbietern gemanaged werden und theoretisch unbegrenzt skalierbar sind, müssen Container Deployments in der Regel manuell konfiguriert und verwaltet werden. 

Ziel der Arbeit ist es, die Performance von Serverless Anwendungen und containerisierten Anwendungen praktisch zu messen und zu vergleichen. Dazu wird eine klassische Beispielanwendung auf beide Arten deployed und der Durchsatz und die anfallenden Kosten mittels Lasttests geprüft.
Es wird Amazon AWS als Testplattform verwendet, der Test-Aufbau soll aber auch die Nutzung anderer Cloud Anbieter ermöglichen. \\

\noindent\textbf{Forschungsfrage} \\
Wie unterscheiden sich der Betrieb von Containerisierten und Serverless (Lambda) Anwendungen in der Cloud (AWS) bezüglich der Performance und Skalierbarkeit?
\end{otherlanguage}