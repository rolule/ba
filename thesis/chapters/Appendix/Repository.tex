\chapter{Repository}
\label{apx:repo}
In diesem Kapitel wird der Aufbau des Code-Repositories erklärt. Es ist unter der URL \url{https://github.com/rolule/ba} auffindbar.

Das Repository ist in zwei Ordner unterteilt. Im Ordner "`thesis"' ist der Latex-Source Code dieser Arbeit zu finden. Der Ordner "`code"' enthält alle für diese Arbeit verwendeten Quellcode-Dateien, Test-Skripte und Test-Artefakte. Zusätzlich ist auf der Wurzel-Ebene des Repositories die PDF-Version dieser Arbeit zu finden. Im folgenden wird sich ausschließlich auf den Ordner "`code"' bezogen.

\section{Anwendungs-Quellcode}
\subsection{Container}
Der Quellcode der Container-Anwendung ist im Unterordner \linebreak "`container/notes-express"' zu finden. Er enthält die zum Bauen des Container-Images benötigte Dockerfile, welche mit Hilfe des Shell-Skriptes "`build.sh"' gebaut werden kann. Zum Bauen des Abbildes wird ein laufender Docker-Daemon benötigt. Die Anwendung kann nach dem Bauprozess mit dem Skript "`run.sh"' auf Port 80 gestartet werden. Der Server sollte beim Aufrufen der URL \url{http://localhost/notes} im Web-Browser eine in JSON formatierte Liste von Notizen anzeigen. 

\subsection{Serverless}
Der Quellcode der Serverless-Anwendung ist im Unterordner \linebreak "`serverless/notes-serverless"' zu finden. Zum Deployen und Ausführen der Anwendung ist das Serverless Framework und ein eingeloggter \ac{AWS} Benutzer notwendig. Sie wurde aber so konzipiert, dass sie sich genau so wie der Container verhält. Der Code der einzelnen Funktionen ist in den Dateien "`get.js"', "`list.js"', "`update.js"' und "`create.js"' zu finden. Die Datei "`serverless.yml"' enthält die Konfiguration des Serverless-Frameworks.

\section{Test Dateien}
Der Ordner "`tests"' enthält alle Dateien die zur Ausführung und Auswertung der Tests notwendig sind. Auch die Test-Artefakte sind hier enthalten. Die im folgenden beschriebenen Dateien befinden sich alle innerhalb dieses Ordners. Um die Skripte in diesem Ordner ausführen zu können, werden einige Abhängigkeiten benötigt. Da auch die Load-Generatoren diese benötigen, wurde ein Shell-Skript geschrieben, welches diese installiert und konfiguriert. Es ist unter dem Dateinamen "`setup.sh"' zu finden und kann auf einer \ac{EC2} Instanz ausgeführt werden, um die Tests-Umgebung zu initialisieren.

\subsection{Test Skripte und Konfiguration}
Das Skript "`test.js"' wurde erstellt, um die Tests auf den Load-Generatoren leichter ausführen zu können. Es wurde in Node.js geschrieben und um es auszuführen, muss Node.js und der Node Paket Manager (npm) installiert sein und die Abhängigkeiten mittels des Befehls \textit{npm install} installiert worden sein. Zur einfacheren Ausführung wurde ein Shell-Skript mit dem Namen "`test.sh"' geschrieben, dass die Node-Applikation mit allen übergebenen Parametern aufruft. 
Das Test-Skript dient der Konfiguration und Ausführung jedes einzelnen durchgeführten Tests. Es bittet den Benutzer um eine Eingabe des ausgewählten Services (Lambda oder Fargate), die konfigurierte Größe (128MB, 256MB, 512MB), den Use-Case (a, b, c) und den Test-Strategien (pipe, stress, load, spike). Zusätzlich wird für Fargate die Anzahl der Tasks, also der Container-Instanzen abgefragt. Die Use-Case Skripte befinden sich im Unterordner "`use-cases"' und werden in der von dem Testing-Tool k6 verstandenen Syntax (in JavaScript) geschrieben. Die Konfigurationen der einzelnen Test-Strategien sind im Unterordner "`strategies"' zu finden und werden k6 bei der Ausführung als Optionen übergeben.

Hat der Benutzer alle Angaben gemacht, wird k6 mit den ausgewählten Test-Einstellungen ausgeführt. Nach Terminierung des Tests, werden dem Benutzer die von k6 ermittelten Metriken angezeigt. Darunter ist die durchschnittliche (avg), minimale (min), maximale (max), mediane Antwortzeit (med) und die 0,90 und 0,95 Quantile. Zusätzlich wird die Gesamtzahl aller Requests dieses Test-Durchlaufs und die Anzahl der durchschnittlichen Requests pro Sekunde ausgegeben. Für weitere Metriken wie die Standardabweichung und den Variationskoeffizienten, werden die Python-Analysetools (siehe \ref{subsec:pandas}) verwendet. Die Metriken der maximalen CPU-Auslastung, maximale Funktions-Dauer, maximale Coldstart-Dauer und maximale Nebenläufigkeit werden von der \ac{AWS} CloudWatch Konsole abgelesen.

\subsection{Test-Artefakte}
Die Artefakte der Tests sind nach Service aufgeteilt und in den Ordnern "`fargate"' und "`lambda"' zu finden. Diese sind weiterhin unterteilt in die unterschiedlichen Use-Cases (a, b und c). In den Use-Case Ordnern befinden sich weitere Unterordner, die die die Tests in die ausgewählte Konfigurations-Größe (wie viel RAM/CPU) eingeteilt werden. Bei Fargate wird mit dem Präfix 2x darauf hingewiesen, dass hier zwei Tasks verwendet wurden. Eine Ebene tiefer sind die unterschiedlichen Test-Strategien aufgeführt. In deren Subordnern befinden sich die Test-Artefakte, die als \textit{tar.gz} Archive abgespeichert sind. Der Name setzt sich dabei immer aus dem Datum in umgekehrter Schreibweise (also aus dem 26.02. wird 0226) und der Uhrzeit des Tests ohne Doppelpunkt (also aus 14:03 wird 1403), verbunden mit einem Bindestrich, zusammen (bspw. 0226-1403.tar.gz). Die Archive lassen sich zum Beispiel unter Linux-Systemen mit Hilfe des Befehls \textit{tar xf 0226-1403.tar.gz} entpacken. Ein entpacktes Archiv enthält immer eine gleichnamige \ac{CSV} und JSON Datei. Die JSON Datei enthält eine von k6 erstellte Zusammenfassung des Tests. Die \ac{CSV} Datei enthält alle von jedem \ac{VU} gemessene Metriken, die k6 unterstützt und erlaubt es somit den Testablauf, mit Hilfe der Analyse-Tools, genau zu rekonstruieren. Die Dateien mussten in Archive verpackt werden, da die \ac{CSV} Dateien aufgrund mehrerer Tausend Zeilen sonst zu groß für das Repository geworden wären.

\subsection{Python Analyse} \label{subsec:pandas}
Im Unterordner "`pandas"' befinden sich die Analyse-Tools, die zur Auswertung in dieser Arbeit verwendet wurden. Es handelt sich um einige Python-Skripte, die die Datenanalyse-Bibliothek "`Pandas"' verwenden und die von k6 erstellten \ac{CSV} Dateien als Eingabe erwarten. Für die Auswertung aller relevanten von k6 erfassten Metriken (außer der durchschnittlichen Requests pro Sekunde) inklusive Standardabweichung und Variationskoeffizient kann das Skript "`stats.py"' verwendet werden. Beispielweise kann mit dem Befehl "`python3 stats.py ../lambda/a/128/pipe/0127-1041.csv"' eine Analyse eines ausgeführten Test der Lambda-Anwendung für Use-Case A, die 128MB Variante und einen Pipe-Clean Tests vom 27. Januar um 10:41 Uhr durchgeführt werden. Wird anstatt der konkreten Datei der ganze "`pipe/"' Ordner angegeben, wird die Analyse über alle \ac{CSV} Dateien innerhalb des Ordners durchgeführt. 
Zur Auswertung von Fehlern eines Tests kann das Skript "`errors.py"' verwendet werden. Das Skript "`time.py"' dient zum Plotten des zeitlichen Verlaufs der Antwortzeit im Median. Identisch dazu ist "`vus.py"', welches aber zusätzlich den Verlauf der virtuellen Benutzer anzeigt.

Es ist wichtig, dass die Skripte ausschließlich innerhalb des "`pandas"' Ordners ausgeführt werden, da nur so der Pfad des Skriptes erfasst werden kann. Für die Ausführung ist eine Python3 Virtuelle-Umgebung notwendig und die Pakete "`pandas"', "`matplotlib"' und "`openpyxl"' müssen mittels der Python-Paketmanagers "`pip"' installiert werden (siehe "`setup.sh"' Skript).