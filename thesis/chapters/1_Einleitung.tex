\chapter{Einleitung}
In diesem Kapitel wird die Motivation und das Ziel dieser Arbeit erläutert und auf die Gliederung eingegangen.

\section{Motivation}
Die Nutzung von Services der Public Cloud Anbieter wie Amazon AWS, Microsoft Azure oder Google Cloud Platform ermöglichen es Organisationen, ihre Anwendungen immer schneller, einfacher und kostengünstiger aufzusetzen und zu betreiben. Schon ein einfaches lift-and-shift von lokalen Rechenzentren hin zu virtuell bereitgestellten Instanzen bedeutet einige Vorteile in puncto Agilität, Kosteneinsparungen und Ausfallsicherheit. Durch Containerisierung und Orchestrierungs-Tools wie Kubernetes wurde es möglich, Anwendungen leichter zu skalieren und noch ausfallsicherer zu machen. Doch die von den Anbietern bereitgestellten Services werden immer raffinierter, um das Betreiben von Software noch weiter zu verbessern. Während beim Betrieb von containerisierten Anwendungen noch gewisse Parameter konfiguriert werden müssen, wird bei Serverless aufgestellten Anwendungen der Server an sich weg abstrahiert. Somit muss sich der Programmierer nicht einmal mehr um die Konfiguration eines Servers kümmern, sondern nur seinen Code schreiben, der dann automatisch von einem Server des Cloud Anbieters ausgeführt wird, wenn er soll und nur wenn er soll. Dies verspricht eine theoretisch unendliche Skalierbarkeit, noch einfacheres Deployment und weitere Kosteneinsparungen. Jedoch ist bisher unklar, wie sich die Performance von Serverless-Anwendungen im Vergleich zu containerisierten Anwendungen verhält. 

\section{Zielsetzung}
Ziel der Arbeit ist es, die Performance von containerisierten und Serverless-Anwendungen anhand eines REST-Backends zu vergleichen. Als Cloud-Anbieter wird Amazon AWS verwendet.

Dazu soll eine Beispielanwendung als REST-API für beide Technologien angepasst entwickelt werden und verschiedenen Performanz-Tests unterzogen werden. Des Weiteren werden die Nutzungskosten beider Technologien betrachtet und evaluiert.

\section{Gliederung}
Zunächst wird auf die Grundlagen von Serverless-Funktionen und Containerisierung eingegangen. Als nächstes soll ein Konzept erstellt werden, mit dem die beiden Technologien evaluiert werden können. Schließlich wird die Evaluation praktisch durchgeführt und die Ergebnisse präsentiert.