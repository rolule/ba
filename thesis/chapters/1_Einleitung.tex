\chapter{Einleitung}
In diesem Kapitel wird die Motivation und Zielsetzung dieser Arbeit erläutert und auf die Gliederung eingegangen.

\section{Motivation}
Der Betrieb von Anwendungen in der Cloud erfährt steigende Aufmerksamkeit. Public-Cloud Anbieter wie \ac{AWS}, Microsoft Azure oder \ac{GCP} erlauben es Organisationen, schneller und günstiger ihre Anwendungen zu provisionieren. Schon ein einfaches Lift-and-shift von lokalen Rechenzentren hin zu virtuell bereitgestellten Server-Instanzen bedeutet einige Vorteile in puncto Agilität, Kosteneinsparungen und Ausfallsicherheit gegenüber On-Premises Infrastruktur. Die Einführung von Containerisierung und Orchestrierungs-Tools wie Kubernetes machte es möglich, Anwendungen einfacher zu skalieren und ausfallsicherer zu machen. Doch die von den Anbietern bereitgestellten Services werden immer raffinierter, um das Betreiben von Software weiterhin zu verbessern und zu erleichtern. Während der Betrieb von containerisierten Anwendungen noch eine manuelle Konfiguration gewisser Parameter erfordert, werden Serverless-Anwendungen fast vollständig von den Cloud-Anbietern gemanaged und der Server somit weg abstrahiert. Sie bieten dadurch eine theoretisch unbegrenzte Skalierbarkeit ohne aufwändige Konfigurierung und Verwaltung  on Infrastruktur. Zudem verspricht Serverless durch ein flexibles Kostenmodell, bei dem nur für tatsächlich ausgeführten Code gezahlt wird, weitere Kosteneinsparungen. Allerdings ist bisher noch unklar, wie sich die Performance von Serverless-Anwendungen im Vergleich zu containerisierten Anwendungen verhält. Dieses Problem soll in dieser Arbeit untersucht werden. 

\section{Forschungsfrage und Zielsetzung}
Ziel der Arbeit ist es, die Performance von containerisierten und Serverless-Anwendungen anhand eines REST-Backends zu vergleichen. Als Cloud-\linebreak Anbieter wird \ac{AWS} verwendet. Es soll eine Beispielanwendung als REST-API für beide Technologien angepasst entwickelt werden und verschiedenen Performance-Tests unterzogen werden. Dabei wird für die Serverless-Applikation \ac{AWS} Lambda verwendet und für die containerisierte Anwendung das Orchestrierungs-Tool \ac{ECS}. Im Anschluss wird auf die Kostenmodelle beider Services eingegangen und diese ins Verhältnis gesetzt.
\newpage
Die Forschungsfrage lautet: Wie unterscheidet sich der Betrieb von containerisierten und Serverless-Anwendungen in der Cloud bezüglich ihrer Performance und Kosten am Beispiel eines REST-Backends auf \ac{AWS} Lambda und \ac{ECS}?

\section{Gliederung}
Zunächst wird auf die Grundlagen von Containerisierung und Serverless-Funktionen am Beispiel von \ac{AWS} Lambda eingegangen. Es wird ein Konzept erstellt, mit dem die beiden Technologien evaluiert werden können. Schließlich wird die Evaluation praktisch durchgeführt und die Ergebnisse präsentiert und diskutiert.