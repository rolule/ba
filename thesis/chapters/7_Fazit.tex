\chapter{Zusammenfassung und Ausblick}
Im Rahmen dieser Bachelor-Thesis wurden Serverless-Funktionen mit \ac{AWS} Lambda und Container-Anwendungen am Beispiel von \ac{ECS} mit Fargate auf ihre Performance untersucht und deren Kostenmodelle analysiert. Dabei wurde deutlich, dass beide Services sowohl Vor- als auch Nachteile in Bezug auf die Performance des Backends hatten. Während Fargate eine allgemein bessere Performance und geringere Varianz aufwies, ist diese aber vor allem durch die CPU-Größe des Containers beschränkt. Lambda zeigte in den Tests größere Antwortzeiten als der Container, überzeugte aber in seinem Skalierungsverhalten bei langsamen und schnellen Anstiegen der virtuellen Benutzer. Es konnte jedoch ein Einfluss mehrerer Endpunkte auf die Antwortzeiten der Lambda-Anwendung gezeigt werden. Dagegen hatte bei der Container-Anwendung weder ein schneller Anstieg der Anfragen noch eine erhöhte Anzahl an Endpunkten eine Veränderung der Antwortzeiten zur Folge. Ebenso führte ein größerer Container oder die Nutzung zweier Instanzen nicht zu geringeren Antwortzeiten. Bei der Lambda-Anwendung hatte die Größe der Funktionen jedoch durchaus einen Einfluss auf die Funktionslaufzeit und damit auch auf die Antwortzeiten.
In der Kostenuntersuchung der beiden Services wurde deutlich, dass diese, obwohl sich gewisse Werte im Voraus schätzen lassen, von der konkreten Nutzung der Anwendung abhängig sind. Besonders bei großer kontinuierlichen Last empfiehlt es sich, eine Lambda-Anwendung mit API-Gateway zu vermeiden und Alternativen wie Container zu bevorzugen.

Es gibt viele verschiedene Variablen die Einfluss auf die Performance von Serverless- und containerisierten Anwendungen nehmen und in zukünftigen Studien untersucht werden können. Beispielsweise könnte das in dieser Arbeit genutzte 50ms Delay variiert, oder durch Nutzung echter Services wie z.B. DynamoDB ersetzt werden. In dieser Arbeit wurde sich auf die Performance eines REST-Backends beschränkt. Dabei wurden verschiedene Container und Lambda-Größen untersucht und unterschiedliche Test-Szenarien durchgeführt. Es konnten aber unmöglich alle verschiedenen Konfigurationen betrachtet und evaluiert werden. Beispielsweise lassen sich sowohl die Größen der Lambda-Funktionen als auch die der Container-Instanzen noch auf mehrere Gigabyte erweitern. Des Weiteren lassen sich Container auf viele verschiedene Wege deployen und betreiben, z.B. auf einem selbst gemanagten \ac{EC2} Cluster, mit \ac{ECS}, einem \ac{EKS} Cluster oder mit Elastic Beanstalk. Auch Aspekte der Optimierung und des unter Umständen komplexen Skalierungsverhaltens einer Container-Anwendung können in zukünftigen Performance-Tests betrachtet werden.

Immer häufiger werden auch Mischformen von Containern und Serverless-Funktionen. Beispielsweise bietet \ac{AWS} seit Ende 2020 an, Container über Lambda verfügbar zu machen\cite{noauthor_aws_nodate-1}. Darüber hinaus könnten Container und \ac{FaaS}-Angebote verschiedener Cloud-Anbieter verglichen werden, zum Beispiel von Microsoft Azure oder Google Cloud Platform.

Die in dieser Arbeit durchgeführten Tests waren auf eine einzige Beispielanwendung beschränkt. Da sich aber für jede beliebige Anwendung Performance-Tests durchführen lassen, könnte das für diese Arbeit genutzte Testing-System in Zukunft ausgebaut werden, damit es für alle möglichen Anwendungs-Architekturen nutzbar ist. In das Analyse-System könnten weitere Metriken aus \ac{AWS} CloudWatch, wie z.B. die tatsächlichen Funktions-Kosten, und andere Container-Services wie \ac{EKS} oder Elastic Beanstalk integriert und somit der Vergleich beider Technologien noch weiter verbessert werden. Dadurch könnte das Tool Organisationen praktisch dabei helfen, die Performance und Kosten der Nutzung von Cloud-Services besser zu evaluieren.
